\section*{Data Management Plan}

\paragraph{Data collection and description}
The project will produce a set of compiler and runtime optimizations, a suite of
real-world applications to evaluate the techniques, performance results data, some
documents and publications. As the project description shows, we will modify some
open-source compilers and collect data based on them.

\paragraph{Data quality and integrity}

Data collection during the project will conform to principles outlined in NSF
policies on Responsible Conduct of Research and the Open Government Directive.
To ensure quality and integrity, data collection will be performed according to
domain best practices and data analysis during the project will determine
validity and appropriateness for data archiving. Domain best practices include
clear descriptions of the software version, architecture, and benchmarks used
to collect the data and release the data in compressed formats to save space.

\paragraph{Data storage and organization}
During the project, data will be stored in text files. Electronic files will be
stored in a local server. To achieve redundancy and protect from catastrophic
failure during the project, the Principal Investigator will store the data in an
on-campus cluster every week. Mines campus computing backs up network servers.
Directory and file naming conventions will be established at the beginning of
the project and may incorporate the conventions of type, place, time, and/or
date.

\paragraph{Data archiving and sharing}
The Principal Investigator will manage data access and use during active project
work. Upon completion of the project, the Principal Investigator will evaluate,
distill, and select the data for archiving. These decisions will be based on
publication usage, storage location, space availability, quality, uniqueness,
broader need/usage, relationship to other data sources, reproducibility, sponsor
requirements, privacy, confidentiality and intellectual property rights. For
those data being archived, electronic data will be archived to a computer server
in a local lab.

The Principal Investigator (PI) will continue to use devices and methods similar
to those described during the active project storage phase. The Principal
Investigator will ensure archiving, preservation, and accessibility of the data
for a minimum of 3 years after the project has ended/final publication has
occurred. Access will be provided to Mines faculty, students and, as
appropriate, researchers at other institutions via email requests.

As an additional means of public access and dissemination and as an
institutional record, appropriate sample data, a data product or overall project
description will be provided to the Mines Institutional Repository
(http://publish.mines.edu). This data sample will have associated metadata and
contact information for getting further access to the broader dataset. This
record at the institutional level allows Mines to keep the Mines Data Inventory
up to date.

Paper records, including laboratory notebooks, raw data, tabulations of data and
results, and associated metadata, will be converted to electronic form and
archived with other electronic data collected during the project. The existence
and availability of the data sets will be communicated to sponsors in progress
reports and to research peers at conferences, and they will be referenced in
publications.

\paragraph{Metadata descriptions}
The Principal Investigator will provide metadata typically associated with the
kind of data generation techniques described above. Some metadata may be
generated as part of the data generation process. Other metadata may be
generated by programmatic or manual means. Mines Research Data Services is
available to advise the PI on descriptive metadata requirements.

